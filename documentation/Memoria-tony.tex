\documentclass[a4paper,openright,12pt]{book}
\usepackage[pass]{geometry}
\usepackage[spanish]{babel}
\usepackage[utf8]{inputenc}
\usepackage{fancyhdr}
\usepackage{graphicx} % graficos


% encabezados

% pie de pagina
\lfoot[]{}
\cfoot[]{}
\rfoot[]{}
\renewcommand{\headrulewidth}{0pt}
\renewcommand{\footrulewidth}{0pt}

% primera pagina de un capitulo
\fancypagestyle{plain}{
	\fancyhead[L]{}
	\fancyhead[C]{}
	\fancyhead[R]{\thepage}
	\fancyfoot[L]{}
	\fancyfoot[C]{}
	\fancyfoot[R]{}
	\renewcommand{\headrulewidth}{0pt}
	\renewcommand{\footrulewidth}{0pt}
}

\pagestyle{fancy}
\setlength{\headheight}{16pt} 

\begin{document}

\begin{titlepage}
	\newgeometry{margin=2cm}
	\vspace*{\fill}
	\begin{center}
		\vspace*{-1in}
		\begin{figure}[htb]
			\begin{center}
				\includegraphics[width=8cm]{images/logo.png}
			\end{center}
		\end{figure}
		\vspace*{0.15in}
		Universidad de Murcia\\
		\vspace*{0.15in}
		Facultad de Informática \\
		\vspace*{0.6in}
		\begin{large}
			\textbf{Máster en Nuevas Tecnologías Informáticas}\\
		\end{large}
		\vspace*{0.1in}
		\begin{large}
			Especialidad en Tecnologías Inteligentes y del Conocimiento con Aplicaciones en Medicina\\
		\end{large}
		\vspace*{0.1in}
		\begin{large}
			Trabajo Fin de Máster:\\
		\end{large}
		\vspace*{0.2in}
		\begin{Large}
			\textbf{Gestión de Bases de Conocimiento Clínicas Masivas} \\
		\end{Large}
		\vspace*{0.3in}
		\begin{large}
			Autor: Toni Wang\\
		\end{large}
		\vspace*{0.2in}
		\rule{80mm}{0.1mm}\\
		\vspace*{0.2in}
		\begin{large}
			Supervisado por: \\
			José Manuel Juárez Herrero \\
			Manuel Campos Martínez \\
		\end{large}
		\vspace*{0.9in}
		Convocatoria de septiembre 2015
	\end{center}
	\vspace*{\fill}
\end{titlepage}

% DEDICATORIA


\chapter*{}
\lhead[\thepage]{}
\rhead[]{}

\pagenumbering{Roman} % para comenzar la numeracion de paginas en numeros romanos
\begin{flushright}
\textit{Dedicado a \\
mi familia}
\end{flushright}

\chapter*{Agradecimientos} % si no queremos que añada la palabra "Capitulo"
 
\chapter*{Resumen} % si no queremos que añada la palabra "Capitulo"
\addcontentsline{toc}{chapter}{Resumen} % si queremos que aparezca en el índice

\chapter*{Extended abstract} % si no queremos que añada la palabra "Capitulo"
\addcontentsline{toc}{chapter}{Extended Abstract} % si queremos que aparezca en el índice

\tableofcontents % indice de contenidos

\cleardoublepage
\addcontentsline{toc}{chapter}{Lista de figuras} % para que aparezca en el indice de contenidos
\listoffigures % indice de figuras

\cleardoublepage
\addcontentsline{toc}{chapter}{Lista de tablas} % para que aparezca en el indice de contenidos
\listoftables % indice de tablas


\chapter{Introducción y motivación}\label{cap.intro}
\renewcommand{\headrulewidth}{0.5pt}
\pagenumbering{arabic} % para empezar la numeración con números

\lhead[\thepage]{CAPÍTULO \thechapter. \rightmark}
\rhead[CAPÍTULO \thechapter. \leftmark]{\thepage}
\markboth{INTRODUCCIÓN}{INTRODUCCIÓN}

\cite{Brewster_CLUK03}


\section{Bases de conocimiento}
\lhead[\thepage]{\thesection. Bases de conocimiento}

\section{Sistemas basados en reglas}
\lhead[\thepage]{\thesection. Sistemas basados en reglas}

\section{Mantenimiento de bases de conocimiento}
\lhead[\thepage]{\thesection. Mantenimiento de bases de conocimiento}

\section{Motivación y enfoque del proyecto}
\lhead[\thepage]{\thesection. Motivación y enfoque del proyecto}

\section{Estado del arte}
\lhead[\thepage]{\thesection. Estado del arte}

\subsection{Sistemas de mantenimiento de bases de conocimiento}
\lhead[\thepage]{\thesubsection. Sistemas de mantenimiento de bases de conocimiento}

\subsection{Técnicas de elaboración de reglas clínicas}
\lhead[\thepage]{\thesubsection. Técnicas de elaboración de reglas clínicas}

\subsection{Técnicas de mantenimiento de reglas clínicas}
\lhead[\thepage]{\thesubsection. Técnicas de mantenimiento de reglas clínicas}

\chapter{Objetivos}
\markboth{OBJETIVOS}{OBJETIVOS}

\chapter{Metodologías y herramientas}
\markboth{METODOLOGÍAS Y HERRAMIENTAS}{METODOLOGÍAS Y HERRAMIENTAS}

\chapter{Diseño y resolución}
\markboth{DISEÑO Y RESOLUCIÓN}{DISEÑO Y RESOLUCIÓN}

\chapter{Análisis de la solución propuesta}
\markboth{ANÁLISIS DE LA SOLUCIÓN PROPUESTA}{ANÁLISIS DE LA SOLUCIÓN PROPUESTA}

\chapter{Conclusiones y vías futuras}
\markboth{CONCLUSIONES Y VÍAS FUTURAS}{CONCLUSIONES Y VÍAS FUTURAS}

%\appendix
%\chapter{Más cosas}\label{aped.A}
%\lhead[\thepage]{APÉNDICE \thechapter. \rightmark}
%\rhead[APÉNDICE \thechapter. \leftmark]{\thepage}
%\markboth{MÁS COSAS}{MÁS COSAS}
%Aún faltan cosas por decir.
%
%\chapter{Y más cosas aún}\label{aped.B}
%
%\markboth{Y MÁS COSAS AÚN}{Y MÁS COSAS AÚN}
%Y más cosas aún.

\cleardoublepage
\addcontentsline{toc}{chapter}{Bibliografía}

\lhead[\thepage]{}
\rhead[]{\thepage}

\bibliographystyle{acm} % estilo de la bibliografía.
\bibliography{bibliografia.bib} %

\end{document}

%%%%%%%%%%%%%%%%%%%%%%%%%%%%%%%%%%%%%%%%%%%%%%%%
% ABSTRACT
%%%%%%%%%%%%%%%%%%%%%%%%%%%%%%%%%%%%%%%%%%%%%%%%
\chapter*{Extended Abstract} % si no queremos que añada la palabra "Capitulo"
\addcontentsline{toc}{chapter}{Extended Abstract} % si queremos que aparezca en el índice
\markboth{Extended Abstract}{Extended Abstract} % encabezado

In the last decade it takes experiencing an increase in the acceptance and use of computers and technology in society. Services like Google, Twitter or Facebook are used en masse by a large number of people. Companies such as Apple, Yahoo or Microsoft are known by most members of today's society. Computer Science is present in every aspect of the lives of people. From research in health care or tourism, to social relationships and even the simple act of food purchase are supported by computers and Computer Science. The Computer Science has a great role in our society wich is gaining more importance and this is a obvious fact.

But Computer Science and their bases have always been unknown to the vast majority of people. It is common to find people that despite daily use a computer at work or with their smartphone, they are unaware of how computers or smartphones work. The computer was created as a tool to solve problems and make people life easier. But it is also important that its performance is understood for everyone to solve every time more complex problems.

Since the early years of computing, many computers scientist have try to extend their knowledge among members of society. Especially among children. It will be the this children who will form the next generation of computer scientits and developers. But they will also form the next generation of doctors, engineers, teachers, fathers and mothers.

Jeannette M. Wing explains in \cite{Wing3717} "Computational thinking is taking un approach to solving problems, designing systems and understanding human behaviour that draws on concepts fundamental to computing". 

It is imperative that computational thinking among younger members of society extends. It has been shown in several studies to learn programming at early ages improves concentration and abstraction, greater autonomy and a notable increase in creativity is achieved. So it is important to enhance computational thinking at an early age so that children develop these skills since childhood.

It exists global projects as Scratch or RoboMind Academy that they are seeking just this goal. Computational thinking training among students in middle school and high school. There are also projects like robot Moway or the Lego Mindstorm NXT and EV3 robotics kits, that they are using robotics to attract youngsters to learn by playing. 

In addition, online platform as CodeHS or Code.org have set a goal to extend the programming among school students. They teach teachers to code for later, they can teach their students how make a program since the beginning and how to code. In short, CodeHS and Code.org are extending computational thinking.

The platform Descubre la programación (it means Discover the programming) is an online platform which aims to teach programming to students in middle school and hifh school at Murcia (Región de Murcia). Descubre la programación provides an online editor to code, challenges and a social network to share programs that are created and learn from peers. It also aims to use the platform to expand programming in schools, providing different tools to track student progress by teachers.

The Descubre la programación language is iJava. iJava is a Java based language with the same sintax but iJava allows imperative programming. iJava gives to developer library functions such as drawing functions, math functions and input and output functions.

In this final degree project we try to bring together all the ideas into a tool to learn programming. It had been created a simulation of a robot that seeks to make programming an attractive and entertaining activity. This simulator will be integrated into the platform Descubre la programación (Discover the programming). In this way, the simulator will be available among students of middle school and high school in Región de Murcia.

So, it joins the idea of teaching programming in an online environment and to use robotics as a way to children entertainment. Improve the childrens computational thinking by learning to code playing. Also, Descubre la programación is an online and free environment. This allows to get anyone to learn programming, anytime and anywhere. 


The simulated robot have similar physical characteristics with robot Moway: two wheels that move independently, a set of external sensors detecting collisions and a lower set of visual sensors to be used for detecting lines painted on ground.




Thus, it is intended that the student invent programs that move the robot randomly, avoiding collisions or following lines. For this purpose, a library of functions that control the behavior of the robot in a transparent way the child is provided. Basic functionality is provided, thereby seeking students to develop within Descubre la programación more complex behaviors themselves.

The key point of this project is to get the student interested in programming from a different point of view that already exists in Descubre la programación. In addition, children will develop its computational thinking and learn to code playing with robots.



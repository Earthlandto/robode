%%%%%%%%%%%%%%%%%%%%%%%%%%%%%%%%%%%%%%%%%%%%%%%%
% ABSTRACT
%%%%%%%%%%%%%%%%%%%%%%%%%%%%%%%%%%%%%%%%%%%%%%%%
\chapter*{Extended Abstract} % si no queremos que añada la palabra "Capitulo"
\addcontentsline{toc}{chapter}{Extended Abstract} % si queremos que aparezca en el índice
\markboth{Extended Abstract}{Extended Abstract} % encabezado

In the last decade it takes experiencing an increase in the acceptance and use of computers and technology in society. Services like Google, Twitter or Facebook are used en masse by a large number of people. Companies such as Apple, Yahoo or Microsoft are known by most members of today's society. Computer Science is present in every aspect of the lives of people. From research in health care or tourism, to social relationships and even the simple act of food purchase are supported by computers and Computer Science. The Computer Science has a great role in our society wich is gaining more importance and this is a obvious fact.

But Computer Science and their bases have always been unknown to the vast majority of people. It is common to find people that despite daily use a computer at work or with their smartphone, they are unaware of how computers or smartphones work. The computer was created as a tool to solve problems and make people life easier. But it is also important that its performance is understood for everyone to solve every time more complex problems.

Since the early years of computing, many computers scientist have try to extend their knowledge among members of society. Especially among children. It will be the this children who will form the next generation of computer scientits and developers. But they will also form the next generation of doctors, engineers, teachers, fathers and mothers.

Jeannette M. Wing explains in \cite{Wing3717} "Computational thinking is taking un approach to solving problems, designing systems and understanding human behaviour that draws on concepts fundamental to computing". 

It is imperative that computational thinking among younger members of society extends. It has been shown in several studies to learn programming at early ages improves concentration and abstraction, greater autonomy and a notable increase in creativity is achieved. So it is important to enhance computational thinking at an early age so that children develop these skills since childhood.

It exists global projects as Scratch or RoboMind Academy that they are seeking just this goal. Computational thinking training among students in middle school and high school. There are also projects like Moway or Lego Mindstorm NXT and EV3 that they are using robotics to attract youngsters to learn by playing. 

In addition, online platform as CodeHS or Code.org have set a goal to extend the programming among school students. They teach teachers to code for later, they can teach their students how make a program since the beginning and how to code. In short, CodeHS and Code.org are extending computational thinking.

The platform Descubre la programación (it means Discover the programming) is an online platform which aims to teach programming to students in middle school and hifh school at Murcia (Región de Murcia). Descubre la programación provides an online editor to code, challenges and a social network to share programs that are created and learn from peers. It also aims to use the platform to expand programming in schools, providing different tools to track student progress by teachers.



In this project we try to bring together all the ideas into a tool to learn programming. You have created a simulation of a robot programming that seeks to make an attractive and entertaining. This simulator will be integrated into the platform Discover programming, allowing available among students of Elementary and Secondary Education of the Region of Murcia.

So, it joins the idea of teaching programming in an online environment, to use robotics as a collection point and entertainment. Learning to program playing. Also, it is an online and free environment is to get anyone to learn programming, anytime and anywhere.

The robot being simulated to have similar physical characteristics of the robot Moway two wheels that move independently, a set of external sensors detecting collisions and a lower set of sensors (optorreflectivos) to be used for detecting lines painted on soil.

Thus, it is intended that the student invent programs that move the robot randomly, avoiding collisions or following lines. For this purpose, a library of functions that control the behavior of the robot in a transparent way the child is provided. Basic functionality is provided, thereby seeking students to develop within Discover more complex behaviors themselves.

The key point of this project is to get the student interested in programming from a different point of view that already exists in Discover. Develop its computational thinking and learn to program playing.

%Desde los primeros años de la informática se ha buscado extender este conocimiento entre los miembros de la sociedad. Especialmente entre los más pequeños, que formarán las nuevas generaciones de informáticos, pero también de médicos, ingenieros y profesores. 

%Es imprescindible que se extienda el \emph{pensamiento computacional} entre los miembros más jóvenes de la sociedad. Se ha demostrado en varios estudios que aprender programación en edades tempranas mejora la capacidad de concentración y abstracción, se consigue mayor autonomía y un notable aumento en la creatividad.

%Proyectos como Robomind o Scratch buscan justo esto, entrenar el pensamiento computacional entre alumnos de educación Primaria y Secundaria. También existen proyectos como Moway o Lego Mindstorm, que utilizan la robótica para atraer a los más pequeños a aprender jugando. 

%Descubre la programación es una plataforma online cuyo objetivo es enseñar a programar a los alumnos de Educación primaria y Secundaria de la Región de Murcia. Descubre proporcionando un editor online, retos y una red social para compartir los programas que se crean y aprender de los compañeros. También tiene la finalidad de que se use la plataforma para extender la programación en los colegios, proporcionando diferentes herramientas para seguir el progreso del alumno. 

%En este proyecto se intenta aglutinar todas las ideas en una herramienta para aprender a programar. Se ha creado un simulador de un robot que busca hacer la programación una actividad atractiva y entretenida. Este simulador se integrará en la plataforma Descubre la programación, permitiendo así que esté disponible entre los estudiantes de Educación Primaria y Secundaria de la Región de Murcia.

%Así, se une la idea de enseñar a programar en un entorno online, a la de usar la robótica como punto de captación y entretenimiento. Aprender a programar jugando. También, que sea en un entorno online y gratuito, se pretende conseguir que cualquiera pueda aprender a programar, a cualquier hora y en cualquier lugar.

%El robot que se simula tendrá unas características físicas similares a las del robot Moway: dos ruedas que se mueven de manera independiente, un conjunto de sensores externos que detectan colisiones y otro conjunto de sensores inferiores (optorreflectivos) que se usarán para detectar lineas pintadas en el suelo.

%De esta manera, se busca que el alumno invente programas que muevan al robot de manera aleatoria, evitando colisiones o siguiendo lineas. Para ello, se proporciona una librería de funciones que permiten controlar el comportamiento del robot de manera transparente al niño. La funcionalidad aportada es básica, buscando de esta manera que los alumnos desarrollen dentro de Descubre comportamientos más complejos por si mismos.

%El punto clave de este proyecto es conseguir que el alumno se interese por la programación desde un punto de vista diferente al que ya existe en Descubre. Que desarrolle su pensamiento computacional y que aprenda a programar jugando.



\appendix

%%%%%%%%%%%%%%%%%%%%%%%%%%%%%%%%%%%%%%%%%%%%%%%%
% Apendice A
%%%%%%%%%%%%%%%%%%%%%%%%%%%%%%%%%%%%%%%%%%%%%%%%
\chapter{Edad escolar en diferentes Sistemas Educativos}\label{anexo:edad-educacion}

Incluso en la Unión Europea, los Sistemas Educativos difieren en la edad de escolarización de los estudiantes. Por ello, se ha confeccionado una tabla que intenta resumir la edad estándar en la que un niño está escolarizado durante las distintas etapas educativas\footnote{Se hablará de las etapas en las que el Sistema Educativo referido está bajo la responsabilidad del Ministerio de Educación (o equivalente) del propio país. Esto no excluye a la educación en centros privados.}. La información que se muestra a continuación está basado en \cite{cursos-educacion-europa} y \cite{guide-education-us}.

Se tomarán como referencia los nombres de las etapas del Sistema Educativo Español (Educación Infantil, Primaria, Secundaria y Bachillerato) para mostrar los años que pasan los estudiantes. Con respecto a la selección de países para realizar la comparación, se escogen los más representativos en los que se han realizado los estudios sobre el aprendizaje de la programación y donde más extendido están las plataformas que se mencionan en el presente documento. 


\begin{table}[!ht]
	\begin{centering}
		\begin{tabular}{c|c|c|c|c}
\emph{País} & Infantil & Primaria & Secundaria & Bachillerato\\
\hline
\emph{España} & 0-6 & 6-12 & 12-16 & 16-18\\
\emph{Estados Unidos} & 3-6 & 6-10 & 10-14 & 14-18\\
\emph{Reino Unido} & 2-5 & 5-11 & 11-16 & 16-18\\
\emph{Alemania} & 0-6 & 6-10 & 10-16 & 16-19\\
\emph{Francia} & 2-6 & 6-11 & 11-16 & 16-18\\
\emph{Bélgica} & 0-2.5/3 & 2.5/3-6 & 6-12 & 12-18\\
\emph{Irlanda} & 4-6 & 6-12 & 12-15 & 15-19\\
\end{tabular}
	\caption{Comparativa de edades de escolarización en diferentes Sistemas Educativos con respecto a las etapas del Sistema Educativo Español.}
		\label{tab:comparativa-tecnicas}
	\end{centering}
\end{table}

En el caso del Sistema Educativo Americano, existen muchas vías en la formación de un niño, como bien detalla A. Corsi-Bunker en \cite{guide-education-us}. Dependiendo de si se escoge una vía privada o dependiendo del estado, los años pueden variar. Aún así, en la figura \ref{tab:comparativa-tecnicas} se muestra el modelo estándar (sistema K-12).\footnote{Para más información sobre el Sistema Educativo Americano recomiendo que se acuda directamente a la fuente: U.S. Department of Education, National Center for Education Statistics (\url{http://nces.ed.gov})}.

\begin{figure}[!ht]
	\begin{centering}
		\includegraphics[width=0.75\textwidth]{images/education-usa.png}
			\caption{Tabla que muestra los típicos patrones de progresión en el Sistema Educativo Americano. Fuente: U.S. Department of Education, National Center for Education Statistics, Annual Reports Program. Obtenido de \url{http://nces.ed.gov/programs/digest/d11/figures/fig_01.asp}}
				\label{fig:education-usa}
	\end{centering}
\end{figure}





%%%%%%%%%%%%%%%%%%%%%%%%%%%%%%%%%%%%%%%%%%%%%%%%
% Apendice B
%%%%%%%%%%%%%%%%%%%%%%%%%%%%%%%%%%%%%%%%%%%%%%%%
\chapter{Lenguaje Logo}
\label{anexo:logo-lenguaje}

A continuación se realizará una revisión de las principales características del lenguaje Logo para tener una visión general de su funcionamiento y sintaxis. Este análisis se basa en el trabajo de \cite[p.274-305]{feurzeig1969programming}. Si el lector quiere ampliar información sobre el lenguaje Logo, puede consultar \cite{friendly2014advanced} o \cite{logo-resources}.


\section{\emph{words}, \emph{sentences}, operaciones y comandos}

Hay dos tipos de datos principales: \emph{words} y \emph{sentences}. \emph{words} está formado por 0 o más caracteres, sin espacios. Algunos ejemplos son: "SUN", "CAT", "5",  "97&!", "" (cadena vacía). El tipo de datos \emph{sentences} se forma por un conjunto de \emph{words} separado por espacios, como por ejemplo "HELLO WORLD" o "3 + 2 = 5". Un numero es una \emph{word} pero conpuesto únicamente por dígitos.

En el lenguaje Logo existen operadores para tratar con los tipos \emph{words} y \emph{sentences}. También se pueden encontrar una serie de operaciones predefinidas en el lenguaje que ofrecen funcionalidad extra, como controlar la entrada/salida de un programa, saber la fecha o manejo de \emph{words} y \emph{sentences}. En la tabla \ref{tab:logo-operaciones} vemos un resumen de las operaciones elementales del lenguaje Logo, el número de argumentos que requiere y un ejemplo de entrada con su salida correspondiente.


\begin{table}[!ht]
	\begin{centering}
		\begin{tabular}{c|c|c|c}
\emph{Operador} & N. argumentos & Entrada & Salida\\
\hline
\texttt{FIRST}	& 1 & "CAT" & "C"\\
- & - & "CAT AND DOG" & "CAT"\\
\texttt{LAST} & 1 & "CAT" & "T"\\
- & - & "CAT AND DOG" & "DOG"\\
\texttt{BUTFIRST} &1& "CAT" & "AT"\\
- & - & "CAT AND DOG" & "AND DOG"\\
\texttt{BUTLAST} & 1 & "CAT" & "CA"\\
- & - & "CAT AND DOG" & "CAT AND"\\
\texttt{COUNT} & 1 & "CAT" & "3"\\
- & - & "CAT DOG" & "2"\\
\texttt{WORD}& 2 & "CAT" "DOG" & "CATDOG"\\
\texttt{SENTENCE}& 2 & "CAT" "DOG" & "CAT DOG"\\
- & - & "CAT " "DOG" & "CAT DOG"\\
\texttt{PRINT} & 1 & "CAT" & CAT\\
\texttt{TIME} & 0 & - & "5:42 PM"\\
\texttt{DATE} & 0 & - & "1/31/2016"\\
\texttt{SUM} & 2 & "-5" "5" & "0"\\
\texttt{DIFFERENCE} & 2 & "5" "5"& "0"\\
\texttt{MAXIMUN} & 2 & "-5" "5" & "5"\\
\texttt{MINIMUN} & 2 & "3" "5"& "3"\\
\end{tabular}
	\caption{Resumen de las operaciones elementales que ofrece el lenguaje Logo.}
		\label{tab:logo-operaciones}
	\end{centering}
\end{table}

Si se intentara ejecutar la operación \texttt{WORD} "CAT " "DOG", se produciría un error al ser uno de sus parámetros un tipo \emph{sentence} y no \emph{word} (el primer argumento contiene un espacio). De igual manera, las operaciones \texttt{SUM}, \texttt{DIFFERENCE}, \texttt{MAXIMUN} y \texttt{MINIMUN} deben recibir argumentos numéricos o se produciría un error.

También es posible encadenar varios operadores. De esta manera, la cadena de instrucciones \texttt{FIRST OF BUTFIRST OF "CAT"} devolverá el \emph{word} "A".


\section{Variables, comentarios y definiendo nuevas operaciones}
\label{sec:logo-variables}

Para definir variables basta con usar el operador \texttt{MAKE} seguido de \emph{"}(comilla doble), el nombre de la variable y el valor que se quiere asignar. Para hacer uso de la variable, utilizamos el símbolo \emph{:}(dos puntos) seguido del nombre de la variable. Para hacer comentarios se utiliza el símbolo \emph{;}(punto y coma). Podemos ver un ejemplo en el código \ref{code:logo-nueva-variable}.

\begin{lstlisting}[language={Logo}, label={code:logo-nueva-variable}, caption={Ejemplo de definición de nuevas variables en el lenguaje Logo.}]
MAKE "ADULTO 18

PRINT :ADULTO  ; salida -> 18
\end{lstlisting}

También es posible crear nuevas operaciones. Se utilizan las palabras reservadas \texttt{TO} y \texttt{END} para marcar el inicio y final de la definición, respectivamente. A continuación es necesario declarar el nombre con el que se definirá la operación que se está creando (el cual no puede coincidir con ninguna palabra reservada en el lenguaje, como es común en la mayoría de lenguajes de programación). Es opcional la declaración de parámetros que se pasaran a la operación. Por último, y antes de la palabra \texttt{END}, se incluirán las instrucciones que realizará la operación.

En el código \ref{code:logo-nuevas-operaciones} se puede ver un ejemplo de definición de nuevas operación.

\begin{lstlisting}[language={Logo}, label={code:logo-nuevas-operaciones}, caption={Definición de nuevas operaciones en el lenguaje Logo.}]
TO SALUDAR
	MAKE "SALUDO "HELLO WORLD"
	OUTPUT SALUDO
END

PRINT SALUDAR  ;  salida -> HELLO WORLD

TO PENULTIMO :algo
	OUTPUT FIRST OF  LAST :algo
END

PRINT PENULTIMO OF "CAT BIRD DOG" ;  salida  ->  BIRD

PRINT PENULTIMO OF "CAT" ;  salida  ->  A
\end{lstlisting}


\section{Operadores aritméticos}

A parte de los operadores \texttt{SUM} y \texttt{DIFFERENCE} antes mencionados, Logo también otorga la posibilidad de utilizar los operadores aritméticos clásicos. La única condición es que la salida generada debe ser tratada (\texttt{PRINT}) o almacenada (en variables, como veremos en la sección siguiente) de alguna manera. Los operadores más importantes son \texttt{+}, \texttt{-}, \texttt{*}, \texttt{/} y \texttt{sqrt}.

\begin{lstlisting}[language={Logo}, label={code:logo-operadores-aritmeticos}, caption=Ejemplo de uso de operadores aritméticos en el lenguaje Logo.]
PRINT 2*3  ;  salida  -> 6

MAKE "RUEDAS  4

PRINT :RUEDAS ;  salida  -> 4
\end{lstlisting}


\section{Operadores \texttt{IF} y \texttt{REPEAT}}

Se pueden crear operaciones condicionales utilizando la palabra clave \texttt{IF} seguida de una sentencia y una lista de órdenes a ejecutar si se evalúa dicha sentencia a verdadero. La lista de órdenes se declara entre corchetes (\texttt{[]}) y para la sentencia se pueden utilizar los operadores infijos \texttt{=}, \texttt{>} y \texttt{<}. En el código \ref{code:logo-if} podemos ver un ejemplo de uso.

\begin{lstlisting}[language=Logo,label={code:logo-if}, caption=Condiciones en el lenguaje logo con el operador \texttt{IF}.]
MAKE "PETALOS 4

IF :PETALOS > 3 [ PRINT "TREBOL DE LA BUENA SUERTE" ]
\end{lstlisting}

Para repetir una serie de instrucciones, Logo ofrece el comando \texttt{REPEAT}, que va seguido de un valor numérico y la lista de instrucciones. El valor numérico (que puede ser parametrizado) es el número de veces que se repetirá en bucle las instrucciones. En el código \ref{code:logo-repeat} vemos un ejemplo de este comando.


\begin{lstlisting}[language=Logo,label={code:logo-repeat}, caption=Condiciones en el lenguaje logo con el operador \texttt{IF}.]
REPEAT 10 [ PRINT "HOLA MUNDO" ]
\end{lstlisting}





%%%%%%%%%%%%%%%%%%%%%%%%%%%%%%%%%%%%%%%%%%%%%%%%
% Apendice C
%%%%%%%%%%%%%%%%%%%%%%%%%%%%%%%%%%%%%%%%%%%%%%%%


\chapter{Programando el movimiento de Turtle en el lenguaje Logo}
\label{anexo:logo-turtle-lenguaje}

A parte de toda la funcionalidad que ofrece Logo como lenguaje, Turtle aporta una librería para poder controlar a nuestra criatura que pintará la pantalla. Ésta suele ser tradicionalmente representada con una tortuga o una flecha. Principalmente, se podrá controlar el movimiento de nuestra tortuga y el giro de ésta. También se le podrá ordenar que deje de pintar para permitir movimiento por la pantalla sin rastro. La mayoría de estos comandos pueden abreviarse para simplificar la escritura.

El resumen sobre la funcionalidad de Turtle que se muestra en las secciones siguientes es un compendio formado a partir de \cite{logo-turtle-lenguaje}, \cite{turtle-academy} y \cite{abelson1980disessa}.


\section*{Moviendo a Turtle}

La tortuga siempre se moverá según el eje de coordenadas. El punto (0, 0) se encontrará normalmente en medio de la pantalla y es la posición inicial de la tortuga. El eje de coordenadas \texttt{y} crece positivamente en dirección Norte (hacia arriba).

Los comandos \texttt{forward} y \texttt{barckward} mueven la tortuga hacia delante y atrás, respectivamente, según la posición en la que esté mirando. Los comandos \texttt{setx}, \texttt{sety} y \texttt{setxy} se utilizan para establecer la tortuga en el eje de coordenadas \texttt{x}, \texttt{y} o en ambos, respectivamente.

La orden \texttt{home} devuelve a la tortuga a la posición (0, 0) de la pantalla. Adicionalmente, se puede ocultar o mostrar (si ya está oculta) la tortuga con las órdenes \texttt{showturtle} y \texttt{hideturtle}.

Otra función muy importante de Turtle, es la opción de girarla. Podemos decidir el giro con las instrucciones \texttt{right} y \texttt{left} y el ángulo de giro (el valor tiene que ser un número positivo). Un ángulo de giro de 380º sería equivalente a girar la tortuga 20º.


\section*{Pintando con Turtle}

Hasta ahora, cualquier desplazamiento de Turtle por la pantalla dejaba una linea o rastro. Con la orden \texttt{penup}, la tortuga dejará de pintar hasta que se ejecute la orden \texttt{pendown}. También se puede ejecutar la instrucción \texttt{clearscreen} que limpiará la pantalla de cualquier rastro dejado por la tortuga.

La instrucción \texttt{label} acepta un argumento y mostrará por pantalla su contenido. El argumento debe ser de tipo \emph{word} o \emph{sentence}\footnote{Para más información sobre los tipos de datos \emph{words} y \emph{sentences}, como se formar y cual es su funcionamiento, se puede consultar la sección \ref{sec:logo-variables}.}.

\section*{Resumen de instrucciones de Turtle}


En la tabla \ref{tab:turtle-lenguaje} se muestra un resumen de las órdenes que puede recibir Turtle.

\begin{table}[!ht]
	\begin{centering}
		\begin{tabular}{c|c|l}
Orden & Abreviatura & Argumentos\\
\hline
forward & fd & Longitud del movimiento\\
backward & bk & Longitud del movimiento\\
home & - & -\\
setx & - & Nueva coodenada X\\
sety & - & Nueva coodenada Y\\
setxy & - & Nueva coodenada X e Y\\
right & rt & Ángulo en el sentido de las agujas del reloj\\
left & lt & Ángulo en el sentido contrario a las agujas del reloj\\
penup & pu & -\\
pendown & pd & -\\
clearscreen & cs & -\\
label & - & Una \emph{word} o \emph{sentence}\\
showturtle & st & -\\
hideturtle & ht & -\\
\end{tabular}
	\caption{Resumen de órdenes en Turtle usando el lenguaje Logo.}
		\label{tab:turtle-lenguaje}
	\end{centering}
\end{table}



%%%%%%%%%%%%%%%%%%%%%%%%%%%%%%%%%%%%%%%%%%%%%%%%
% Apendice D
%%%%%%%%%%%%%%%%%%%%%%%%%%%%%%%%%%%%%%%%%%%%%%%%
\chapter{Programación con Robomind Academy}
\label{anexo:programacion-robomind}

En Robomind se plantea la programación de un robot simulado que está dentro de un escenario. Este escenario es una cuadrícula por la que se mueve el robot. El robot puede interactuar con el escenario pintando o limpiando el suelo, recogiendo y dejando objetos, o comprobando el estado de su casillas adyacentes (si hay algún objeto, obstaculo o está pintado).

A continuación vamos a enumerar las funciones básicas para controlar el movimiento o acciones del robot de Robomind\footnote{Para una información más detallada se puede consultar \url{https://www.robomindacademy.com/go/robomind/help}.}. Adicionalmente, Robomind también permite usar definición de variables y nuevas funciones, estructuras de control y bucles, tipícos de cualquier lenguaje de programación moderno.

\begin{itemize}
\item Moviendo el robot:
	\begin{itemize}
	\item  \texttt{forward} o \texttt{forward(n)}: Mover el robot 1 o \texttt{n} pasos hacía el frente.
	\item \texttt{backward} o \texttt{backward(n)}: Mover el robot 1 o \texttt{n} pasos hacía el atrás.
	\item \texttt{left} o \texttt{left(n)}: Girar el robot 90º hacía la derecha, 1 o \texttt{n} veces.
	\item \texttt{right} o \texttt{right(n)}: Girar el robot 90º hacía la derecha, 1 o \texttt{n} veces.
	\end{itemize}
\item Interactuando con el escenario:
	\begin{itemize}
		\item \texttt{pickUp}: Coger el objeto que se encuentra delante del robot.
		\item \texttt{putDown}: Dejar el objeto justo delante del robot.
		\item  \texttt{eatUp}: \emph{Comer} el objeto en frente del robot. Implica destruirlo y no poder volver a interactuar con él.
		\item \texttt{paintWhite}:  Pinta un rastro blanco por las cuadrículas por las que se desplazará el robot.
		\item \texttt{paintBlack}: Pinta un rastro negro por las cuadrículas por las que se desplazará el robot.
		\item  \texttt{stopPainting}:Deja de pintar.
	\end{itemize}	
\item El robot puede comprobar, de manera independiente, el estado de las casillas adyacentes a sí mismo, excepto hacia atrás. Para representar cada dirección (\emph{front}, \emph{left} y \emph{right}), se usará una \emph{X}. Todas estas instrucciones devuelven un valor \emph{booleano}.
	\begin{itemize}
		\item \texttt{XIsClear}:  Comprueba si la casilla \emph{X} está vacía.
		\item \texttt{XIsObstacle}:  Comprueba si en la casilla \emph{X} hay un obstáculo, generalmente una pared o un objeto que impide su avance.
		\item \texttt{XIsBeacon}:  Comprueba si en la casilla \emph{X} hay un objeto (con el que podrá interactuar).
		\item \texttt{XIsWhite}:  Comprueba si la casilla \emph{X} está pintada de blanco.
		\item \texttt{XIsBlack}:  Comprueba si la casilla \emph{X} está pintada de negro.
	\end{itemize}	
\end{itemize}


%%%%%%%%%%%%%%%%%%%%%%%%%%%%%%%%%%%%%%%%%%%%%%%%
% Apendice D
%%%%%%%%%%%%%%%%%%%%%%%%%%%%%%%%%%%%%%%%%%%%%%%%
\chapter{Programando con Scratch}
\label{anexo:scratch-funcionamiento}

En general, la plataforma Scratch tiene tres grandes componentes:
\begin{enumerate}
	\item \textbf{Un editor visual} que presenta un conjunto de bloques cuya forma ofrece pistas de como pueden encajar los diferentes bloques.
	\item \textbf{Un interprete} que puede traducir y ejecutar el flujo de código que forman los diferentes bloques unidos entre sí.
	\item \textbf{Una interfaz} que muestra la salida (tanto de texto como gráfica) del código ejecutado por el interprete. También permite controlar la entrada del ratón, de las teclas, etc.
\end{enumerate}

En la figura \ref{fig:scratch-example} podemos ver un ejemplo del editor y la interfaz así como de un programa de ejemplo que mueve al personaje seleccionado por la pantalla, cambiado su color.

\begin{figure}[!ht]
	\begin{centering}
		\includegraphics[width=1\textwidth]{images/scratch-example.png}
			\caption{Ejemplo de la interfaz de usuario y el editor de código que ofrece Scratch.}
				\label{fig:scratch-example}
	\end{centering}
\end{figure}

Cuando se programa con Scratch, la meta final es la de animar uno o más elementos por la pantalla. Estos elementos son los \emph{sprites} (imágenes), los cuales vienen predefinidos en una galería o pueden ser proporcionados por el propio usuario. Por tanto, los bloques irán enfocados a otorgar movimiento al \emph{sprite} en cuestión.

En cuanto a los tipos de bloques, actualmente están divididos en 11 categorías distintas:
\begin{itemize}
	\item \textbf{Movimiento:} Incluye funciones para mover, girar o establecer la posición y ángulo de giro del \emph{sprite}.
	\item \textbf{Apariencia:} Esta sección contiene funciones tanto para modificar la apariencia del \emph{sprite} como para mostrar mensajes en la pantalla a modo de diálogos de los \emph{sprites}.
	\item \textbf{Sonido:} Funciones para controlar el sonido de la aplicación que se está creando con Scratch. Permite reproducir sonidos precargados en la plataforma o subidos por el programador, diferentes instrumentos o establecer el volumen de la aplicación.
	\item \textbf{Pincel:} Permite pintar el escenario de forma similar a Turlte, dejando un rastro por donde se va desplazando (para más información sobre Turtle, consultar sección \ref{sec:turtle}).
	\item \textbf{Datos:} Creación de variables y listas y modificación de los valores de las mismas.
	\item \textbf{Eventos:} Bloques para detectar y reaccionar a ciertos eventos como el click de un ratón sobre un \emph{sprite} o el teclado.
	\item \textbf{Control:} Incluye bloques con las condiciones y bucles clásicos.
	\item \textbf{Percepción:} Esta sección contiene funciones para detectar eventos de teclado y ratón. También provee funciones para establecer lapsos de tiempo. 
	\item \textbf{Operadores:} Bloques para utilizar operadores matemáticos.
	\item \textbf{Bloques propios} y \textbf{Extensiones}: Scratch permite crear tus propios bloques y usar extensiones integradas con la aplicación como \emph{Lego WeDo} y \emph{PicoBoard}. 
\end{itemize}


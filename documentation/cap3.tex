%%%%%%%%%%%%%%%%%%%%%%%%%%%%%%%%%%%%%%%%%%%%%%%%
% 3: Análisis de objetivos y metodología
%%%%%%%%%%%%%%%%%%%%%%%%%%%%%%%%%%%%%%%%%%%%%%%%
\chapter{Objetivos, metodología y herramientas}
\label{objetivos-metodologia-herramientas}

\section{Análisis de objetivos}
\label{sec:objetivos}

Este Trabajo Fin de Grado consiste en desarrollar un módulo de simulación de un robot para integrarlo en la plataforma \Gls{descubre}(\url{descubre.inf.um.es}).

El simulador se ha denominado \emph{Robode} y éste simulará un robot real con funciones realistas y en un entorno continuo (y no discreto). El simulador dispondrá de una serie de características básicas:
\begin{itemize}
	\item Robode será un robot de dos ruedas.
	\item Cada rueda de Robode se moverá de manera independiente, simulando un motor para cada rueda. 
	\item Robode se encontrará en un circuito finito. 
	\item El circuito tendrá \emph{lineas} pintadas en el suelo, \emph{obstáculos} y \emph{fronteras}.
	\item Las lineas pintadas el suelo no colisionarán con Robode.
	\item Los obstáculos podrán colisionar con Robode, produciendo un efecto en el contacto de ambos cuerpos, desplazándolos según la velocidad y características físicas de cada uno.
	\item Las fronteras serán elementos estáticos del circuito, impidiendo su movimiento o el de Robode a través de las mismas. El circuito tendrá fronteras limitando su superficie.
	\item Dispondrá de 2 sensores que, mirando al suelo, detectarán lineas pintadas y, por tanto, si Robode se encuentra sobre una linea.
	\item Dispondrá de sensores de proximidad delanteros y traseros, que detectarán si se va a producir una colisión con un obstáculo o una frontera.
\end{itemize}


El simulador se integrará con Descubre, una plataforma para aprender programación utilizando el lenguaje iJava\footnote{Para más información sobre la plataforma Descubre y el lenguaje iJava se puede consultar la sección \ref{sec:descubre}.}. Por tanto, será necesario entender el funcionamiento interno de la plataforma y modificarla apropiadamente para poder integrar Robode con Descubre. De está manera, se añadirán las funciones necesarias para poder crear programas en iJava y controlar el robot.


\section{Metodología}
\label{sec:metodologia}

% JS: pag 36 tfg aletea 

{\color{green}
Empty yet.
}


\subsection{Tecnologías}
\label{sec:tecnologias}


Robode utiliza una serie de tecnologías web sobre las que trabaja para poder integrarse en el entorno de Descubre y crear un simulador en un entorno online. A continuación se describirán brevemente estas tecnologías.

\begin{itemize}
	
	\item \textbf{HTML} o \emph{HyperText Markup Language} (lenguaje de marcas de hipertexto)	es un estándar que proporciona la estructura de las páginas web. En concreto, se ha utilizado la última versión de HTML, HTML5, que proporciona elementos como \emph{canvas}, el cual usaremos para dibujar nuestro robot.
	
	\item \textbf{CSS} son las siglas para \emph{Cascading Style Sheets}. Es un lenguaje para definir el estilo (o presentación) de la página web sobre HTML.
	
	\item \textbf{Javascript}, como se le conoce comúnmente, es un lenguaje para programar páginas web que trabaja junto a HTML. Es un lenguaje interpretado por los navegadores, con orientación a objetos basada en prototipos, imperativo, con tipado débil y dinámico. Javascript está estandarizado por ECMA International\footnote{Página oficial de ECMA International (\url{http://www.ecma-international.org}).} y su nombre oficial es ECMAScript. La versión más extendida de Javascript actualmente es ECMAScript 5. Aunque una nueva versión se hizo pública en junio de 2015\cite{ecmascript6}. 
	
\end{itemize}



\section{Herramientas utilizadas}
\label{sec:herramientas}

Las herramientas que se han usado para desarrollar este proyecto se describen a continuación:

\begin{itemize}
	\item \textbf{Atom}\footnote{Página oficial de Atom (\url{https://atom.io}).} como editor de código. Acompañado de paquetes \emph{linter}\footnote{Un \emph{linter} analiza el código escrito en busca de errores tipográficos. Para más información sobre el \emph{linter} de Atom, se puede consultar la página web de la herramienta: \url{https://atom.io/packages/linter}.} para Javascript, HTML y CSS que detectan la mayoría de los errores sintácticos que se producen a la hora de programar. También se han utilizado diferentes paquetes para tabular y organizar el código automáticamente. 

	\item Como herramienta de control de versiones se ha utilizado \textbf{Git}\footnote{Página oficial del proyecto Git (\url{https://git-scm.com}).} y como repositorio \textbf{GitHub}\footnote{Página oficial de GitHub (\url{https://github.com})}.
	
	\item Para las pruebas, ejecución y depuración de la aplicación se ha utilizado tanto \textbf{Google Chrome}\footnote{Página oficial de Google Chrome (\url{https://www.google.es/chrome/browser/desktop/}).} versión 48, como \textbf{Safari}\footnote{Página oficial de Safari (\url{http://www.apple.com/es/safari/}).} versión 9.0.3.
	
	\item Para el desarrollo de esta documentación se ha utilizado \textbf{MacTex} como motor de \emph{Latex}\footnote{Página oficial del proyecto Latex (\url{https://www.latex-project.org}).} y \textbf{TexMaker}\footnote{Página oficial de TexMaker (\url{http://www.xm1math.net/texmaker/}).} para la redacción del documento. También se ha utilizado \textbf{BibTex}\footnote{Página oficial de BibTex (\url{http://www.bibtex.org}).} como gestor bibliográfico.
\end{itemize}



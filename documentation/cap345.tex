%%%%%%%%%%%%%%%%%%%%%%%%%%%%%%%%%%%%%%%%%%%%%%%%
% 3: Análisis de objetivos y metodología
%%%%%%%%%%%%%%%%%%%%%%%%%%%%%%%%%%%%%%%%%%%%%%%%
\chapter{Análisis de objetivos, metodología y herramientas}\label{objetivos-metodologia}

\section{Objetivos}
\label{sec:Objetivos}

Este Trabajo Fin de Grado consiste en desarrollar un módulo de simulación de un robot para integrarlo en la plataforma \Gls{descubre}. Para ello se tendrán que cubrir una serie de subobjetivos que nombraremos a continuación.
\begin{itemize}
\item Estudiar el uso y aplicación de diferentes librerías de físicas para generar el robot.
\item Comprensión de la plataforma Descubre así como su posterior modificación.
\item Creación de un simulador de un robot de dos ruedas y su integración en la plataforma Descubre.
\item Modificación del motor de \gls{ijava} y creación de la \acrshort{API} para poder controlar el robot.
\end{itemize}


\section{Metodología}
\label{sec:metodologia}

% JS: pag 36 tfg aletea




\section{Herramientas utilizadas}
\label{sec:herramientas}





%%%%%%%%%%%%%%%%%%%%%%%%%%%%%%%%%%%%%%%%%%%%%%%%
% 4: Diseño y resolución del trabajo realizado
%%%%%%%%%%%%%%%%%%%%%%%%%%%%%%%%%%%%%%%%%%%%%%%%
\chapter{Diseño y resolución del trabajo realizado}
\label{diseno}



%Decidir que poner, como, en que orden.




%%%%%%%%%%%%%%%%%%%%%%%%%%%%%%%%%%%%%%%%%%%%%%%%
% 5: Conclusiones y vías futuras
%%%%%%%%%%%%%%%%%%%%%%%%%%%%%%%%%%%%%%%%%%%%%%%%
\chapter{Conclusiones y vías futuras}\label{conslusiones}





%explicar en que se diferencia robode con el resto y porque es mejor o que ventajas tiene
%explicar como resuelvo los problemas que se encuentra un estudiante cuando programa (intro-3º parrafo)


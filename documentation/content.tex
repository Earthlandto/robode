%%%%%%%%%%%%%%%%%%%%%%%%%%%%%%%%%%%%%%%%%%%%%%%%
% 1: Introducción
%%%%%%%%%%%%%%%%%%%%%%%%%%%%%%%%%%%%%%%%%%%%%%%%
\pagenumbering{arabic} % para empezar la numeración con números
\chapter{Introducción}\label{introduccion}


%Desde el comienzo de su historia, la informática ha intervenido en cada aspecto científico y social existente. La dependencia de la sociedad en una ciencia tan reciente resalta la importancia de que sea algo comprensible y comprendido para la sociedad misma.

Como dice M. Lemaire\cite{lemaire2014incorporating}, en una sociedad que está incrementando su dependencia a las nuevas tecnologías\footnote{Aquí, con nuevas tecnologías me refiero a productos y proyectos derivados de la Informática y las \acrfull{TIC} como puede ser los proyectos que detallamos en la sección \ref{estado-arte}.}, es imprescindible que las nuevas generaciones desarrollen la habilidad de pensar de manera crítica sobre tecnología.

El \emph{pensamiento computacional}, como se refiere A. Bundy en \cite{bundy2007computational}, afecta a  investigaciones de casi todas las disciplinas, tanto de ciencias como de humanidades. [...] La informática no solo ha permitido que los investigadores puedan hacer nuevas preguntas, sino también ha permitido aceptar nuevos tipos de respuesta. Por ejemplo, preguntas que requieren el procesamiento de una gran cantidad de datos.

Es necesario comprender la informática y desarrollar el \emph{pensamiento computacional}. Se necesitan desarrollar nuevas tecnologías, nuevo \gls{hardware} y \gls{software} que automatice tareas largas, complejas y con una alta cantidad de cómputo. Tareas que no siempre los humanos podemos resolver de manera directa. Para conseguir esto, muchas veces es necesario programar.

Pero aprender a programar es mencionado como uno de los 7 grandes retos de la educación informática \cite{mcgettrick2005grand} y diversos estudios \cite{renumol2009classification} muestran que las principales dificultades para un alumno cuando está en el proceso de aprender a programar son (a) como empezar un programa; (b) comprensión de la sintaxis específica del lenguaje de programación; (c) comprensión de la lógica\footnote{En este caso me refiero a lógica booleana, o también llamada Álgebra de Boole.} y (d) problemas a la hora de depurar el código escrito.

Aunque a primera vista aprender a programar pueda parecer una tarea ardua y compleja, tiene sus ventajas. En el estudio realizado por J. Siegmund y otros en \cite{siegmund2014understanding}, podemos ver como los participantes mientras comprendían, analizaban y buscaban errores en pequeños trozos de código, daban claras muestras de estar desarrollando actividad cerebral en regiones del cerebro relacionadas con el procesado del lenguaje, la atención y la memoria de trabajo.

De la misma manera, estudios realizados en niños \cite{clements1986effects} de entre 6 y 8 años, muestran que estos demostraron mayor capacidad de atención, más autonomía y un mayor placer por el descubrimiento de nuevos conceptos. En la misma linea, un estudio en niños de infantil \cite{logo-geometry} que utilizaban el \Gls{logo} demostró que los mismos obtuvieron mejores resultados en pruebas de razonamiento, matemáticas o resolución de problemas. Otro estudio más reciernte \cite{liao1991effects} demuestra que aprender a programar (independientemente del lenguaje) a una corta edad, potencia la creatividad y la habilidad de aprendizaje.



\section{Situación actual}
\label{sec:situacion-actual}


Actualmente existen muchos proyectos con el fin de introducir la programación como asignatura obligatoria en educación primaria y secundaria. Estos proyectos vienen respaldados por importantes cambios en los planes de estudios de todo el mundo para enseñar programación\cite{cs-education,madrid-programacion,codigo21, guide-england-computing,chicago-cs}, marcando una clara tendencia social de incluir la programación en los currículos académicos.


Uno de los proyectos más importantes y con más repercusión a nivel global es Code.org\cite{code-org}. Propone una serie de herramientas y juegos diseñados especialmente para que los niños aprendan programación mientras juegan con sus personajes favoritos de películas y videojuegos. Todo ello acompañado de una infraestructura para que profesores puedan incluir estos proyectos en las aulas, y padres en sus casas. 
Su proyecto \emph{Hora del código}\cite{hour-of-code} consigue reunir a niños de todas las edades una hora al día para que la inviertan en programar. 
Code.org cuenta con decenas de millones de estudiantes de más de 180 países, disponible en más de 30 idiomas. De manera gratuita, cualquiera puede aprender a programar en eventos que se realizan por todo el mundo (figura \ref{fig:map-hour-code}). Code.org está apoyado por grandes compañías y persanalidades a nivel global como puede ser Microsoft, Google, el Presidente Barack Obama, Mark Zuckerberg, Bill Gates o Walt Disney Company. {\color{red}En la sección \ref{sec:Code.org} se abordará más detalladamente las diferentes formas que tiene Code.org de enseñar a programar.}


\begin{figure}[!ht]
	\begin{centering}
		\includegraphics[width=0.8\textwidth]{images/map-hour-code.png}
			\caption{Mapa de visualización de eventos de llevados a cabo por el proyecto \emph{Hora del código} alrededor del mundo. Actualmente 198,473 en todo el mundo, 1,839 en España. Obtenido de \cite{hour-of-code}.}
				\label{fig:map-hour-code}
	\end{centering}
\end{figure}


A nivel europeo, la \Gls{com-euro}{\color{red}\cite{ec-code-week}} ha promovido durante el año 2015 la \emph{EU Code Week}{\color{red}\cite{code-week}} como parte de su Estrategia para la Educación y la Formación 2020. Este proyecto consistió en eventos de una semana de duración en la que se enseñaba informática y programación en lugares de toda Europa\footnote{En España, se realizaron eventos dentro del marco del proyecto \emph{EU Code Week} en Madrid, Sevilla, Murcia, Asturias, Canarias, Cantabria, Zamora, Cataluña, Ceuta, Badajoz, La Rioja, País Vasco y Valencia.}.



\section{Proyecto Descubre la programación}
\label{sec:descubre}

\Gls{descubre}\cite{descubre} es un proyecto {\color{red}desarrollado por la Facultad de Informática de la Universidad de Murcia} y tiene como objetivo ayudar a los alumnos de secundaria y bachillerato a que desarrollen sus capacidades descubriendo lo que es la informática y aprendiendo a programar. Así como fomentar la inclusión del aprendizaje de la programación en secundaria y bachillerato.

Para ello, en un mismo sitio web, se integra (a) un conjunto de tutoriales de programación (figura \ref{fig:descubre-aprende}); (b) una herramienta que permite programar (figura \ref{fig:descubre-crea}) y realizar ejercicios o retos propuestos (figura \ref{fig:descubre-retos}) y (c) una pequeña red social que permite publicar y compartir con el resto de compañeros los programas realizados (figura \ref{fig:descubre-explora}). Adiccionalmente se puede consultar las estadísticas de aprendizaje (figura \ref{fig:descubre-perfil}) y tiempo dedicado (figura \ref{fig:descubre-estadisticas}) en la plataforma, tanto por el alumno como por el profesor. De esta manera se permite que los profesores puedan utilizar Descubre en las aulas y realizar un seguimiento de la dedicación y progreso del alumno. 

En cuanto a la herramienta para desarrollar programas, el lenguaje utilizado es \Gls{ijava}\cite{sanchez2009ijava} y ha sido desarrollado por J. A. Sánchez Laguna. iJava es un lenguaje imperativo basado en \Gls{java} y comparte su sintaxis, aunque se han eliminado todos los componentes del lenguaje Orientado a Objetos. También incorpora un conjunto reducido de funciones de librería clasificadas en los tres grupos siguientes: númericas, entrad/salida y gráficas. iJava se estudiará más en detalle en el apéndice \ref{anexo:ijava}.

A modo de ilustración, podemos ver el código \ref{code:hello-world}, un programa que imprime por pantalla la cadena "Hello World.". En el código \ref{code:circulos-color-raton} podemos ver un programa un poco más complejo que dibuja círculos de colores en la pantalla según la posición del ratón. En la figura \ref{fig:salida-code-circulos-color-raton} se puede ver la salida de este último programa.


\begin{lstlisting}[caption=Programa básico en iJava imprimiendo por pantalla la cadena "Hello World.".]
void main(){
  print("Hello World.");
}
\end{lstlisting}\label{code:hello-world}


\begin{lstlisting}[language=Java, caption=Programa en iJava que dibuja un circulo de un color diferente según la posición en la pantalla en la que se encuentra el ratón.]
void main() {
  //repetimos en bucle la función 'draw'
  animate(draw);
}

void draw() {
  //coloreamos el circulo según la posición del raton
  fill(mouseX, mouseY, 0); //valores RGB
  //dibujamos una elipse con radio 50
  ellipse(mouseX, mouseY, 50,50);
}
\end{lstlisting}\label{code:circulos-color-raton}


La sintaxis de iJava se parece a muchos de los lenguajes modernos y más utilizados (al ser un subconjunto de la sintaxis de Java), lo cual simplifica el aprendizaje cuando se intenta aprender un nuevo lenguaje. También, gracias a la librería gráfica y matemática, los programas se simplifican mucho en cuanto a complejidad y longitud. De esta manera se consigue aligerar la carga de trabajo que tiene que realizar el alumno para conseguir hacer un programa vistoso, enganchándolo más fácilmente a la actividad de programar.


\begin{figure}[!ht]
	\begin{centering}
		\includegraphics[width=0.7\textwidth]{images/descubre-crea.png}
				\caption{Sección \emph{Crea} del proyecto \Gls{descubre}. Obtenido de \cite{descubre}.}
				\label{fig:descubre-crea}
	\end{centering}
\end{figure}



\begin{figure}[!ht]
	\begin{centering}
		\includegraphics[width=0.5\textwidth]{images/descubre-explora.png}
				\caption{Sección \emph{Explora} del proyecto \Gls{descubre}. Obtenido de \cite{descubre}.}
				\label{fig:descubre-explora}
	\end{centering}
\end{figure}



\begin{figure}[!ht]
	\begin{centering}
		\includegraphics[width=0.5\textwidth]{images/descubre-aprende.png}
				\caption{Sección \emph{Aprende} del proyecto \Gls{descubre}. Obtenido de \cite{descubre}.}
				\label{fig:descubre-aprende}
	\end{centering}
\end{figure}


\begin{figure}[!ht]
	\begin{centering}
		\includegraphics[width=0.5\textwidth]{images/descubre-retos.png}
				\caption{Sección \emph{Retos} del proyecto \Gls{descubre}. Obtenido de \cite{descubre}.}
				\label{fig:descubre-retos}
	\end{centering}
\end{figure}


\begin{figure}[!ht]
	\begin{centering}
		\includegraphics[width=0.5\textwidth]{images/descubre-profile.png}
				\caption{Vista del perfil de usuario del proyecto \Gls{descubre}. Obtenido de \cite{descubre}.}
				\label{fig:descubre-perfil}
	\end{centering}
\end{figure}


\begin{figure}[!ht]
	\begin{centering}
		\includegraphics[width=0.5\textwidth]{images/descubre-statistics.png}
				\caption{Sección de estadísticas del usuario del proyecto \Gls{descubre}. Obtenido de \cite{descubre}.}
				\label{fig:descubre-estadisticas}
	\end{centering}
\end{figure}


\begin{figure}[!ht]
	\begin{centering}
		\includegraphics[width=0.5\textwidth]{images/salida-code-circulos-color-raton.png}
			\caption{Salida del programa escrito en iJava ejecutado por el código \ref{code:circulos-color-raton}.}
				\label{fig:salida-code-circulos-color-raton}
	\end{centering}
\end{figure}





\section{Motivación y enfoque del proyecto}
\label{sec:motivacion}

Las nuevas tecnologías están haciendo que cada vez más gente de todas las edades se interese por la informática, y más concretamente por la programación. Poco a poco la informática deja de ser cosa de un grupo selecto de gente {\red{color}que sabe que una aplicación no es \emph{magia}}.

Como ya hemos comentado anteriormente, existe un movimiento que pretende introducir la informática y la programación en las aulas. Se han demostrado los beneficios de enseñar a programar en una edad escolar temprana.

Es en este momento cuando se hace imprescindible tomar decisiones acertadas. Aprender a programar en edades pre-universitaria debe ser algo accesible a cualquier estudiante, independiente de su condición o los antecedentes del mismo. Igualmente, los conceptos de programación deben ser presentados de manera {\color{red}escalonada y paulatina \cite{fernandez2002analisis}}.

El proyecto se enfocará desarrollando un simulador de robot de dos ruedas en el que los alumnos puedan programar su comportamiento. Se ofrece una librería de funciones para simplificar la interacción con el mismo y conseguir cierta funcionalidad extra que simplifique la tarea de comprensión y usabilidad del simulador.

El simulador se integrará en la plataforma Descubre la pogramación, mencionada en la sección \ref{sec:descubre}. Esto conlleva que el estudiante programará el comportamiento del robot en iJava. El robot estará dentro de un circuito con distintos elementos con los que podrá interactuar. Asimismo, el robot dispondrá de una serie de sensores para poder recibir información del mundo que le rodea.

Al incluir el simulador como un módulo de descubre, se pretende reforzar el esfuerzo por parte de sus creadores de hacer llegar la programación al mayor numero de estudiantes pre-universitarios posible, proponiendo una alternativa más vistosa, que añade una componente más de entretenimiento a la actividad de programar.

Igualmente, los estudiantes que ya están usando la plataforma Descubre, podrán trabajar los conceptos de programación (bucles, condiciones, variables y funciones, entre otras) en un entorno diferente, renovando así su interés en otra actividad diferente.


%% Último párrafo: explicar de que van las siguientes secciones.
En las secciones siguientes se hablará de cuales son las alternativas ya creadas que trabajan en esta linea. También se verá como se ha desarrollado la idea principal y que resultados se han obtenido. Por último, se analizarán los objetivos conseguidos y que ofrece mi propuesta de diferente con los proyectos ya desarrollados.




{\color{red}¿Se puede decir que me he explicado el "enfoque del proyecto"?}



%%%%%%%%%%%%%%%%%%%%%%%%%%%%%%%%%%%%%%%%%%%%%%%%
% 2: Estado del arte
%%%%%%%%%%%%%%%%%%%%%%%%%%%%%%%%%%%%%%%%%%%%%%%%
\chapter{Estado del arte}\label{estado-arte}



A nivel global, actualmente existe una gran cantidad de proyectos con la única intención de enseñar, principalmente a algumos de secundaria y bachillerato, diferentes aspectos de la informática como lo es la programación \cite{code-school,code-org,code-academy}, la robótica \cite{robomind-web,moway} e incluso chips y electrónica con \Gls{arduino}\cite{arduino}. Algunos de estos proyectos llevan décadas activos, como lo es el lenguaje \Gls{logo}\cite{logo} y su proyecto \Gls{turtle}\cite{logo-turtle}
La mayoría de estos proyectos promueven una enseñanza independiente y autodidacta bajo un entorno on-line. De esta manera, el alumno puede aprender a su propio ritmo y desde cualquier parte del mundo.


\section{El lenguaje Logo y el robot Turtle}
\label{sec:Logo}

El lenguaje Logo, basado en \Gls{lisp}, fue diseñado como una herramienta para aprendizaje. Todas sus características -interactividad, modularidad, extensibilidad, flexibilidad en los tipos de datos- persiguen esta meta. El lenguaje Logo fue uno de los primeros proyectos en emprender la difícil tarea de enseñar a programar a niños de primaria.

Como se explica en la página oficial del proyecto Logo\cite{logo}, durante la década de los 70, en el \acrfull{MIT} y diferentes centros de investigación europeos, se llevó a cabo una investigación sobre el uso del \Gls{logo} en un pequeño grupo de alumnos de secundaria. Todo el proceso se ha documentado en diferentes artículos como \cite{feurzeig1969programming} o \cite{pea1984logo}.

El proyecto Turtle es el proyecto más popular del lenguaje Logo. Nació como una criatura robótica que se movía por el suelo y se podía programar solo con 2 instrucciones básicas: \texttt{forward x} y \texttt{right y}, para avanzar \texttt{x} \emph{pasos de tortuga} o girar \texttt{y} grados hacia la derecha, respectivamente.

Combinando estas dos instrucciones de lo más simples, nuestro robot tortuga puede realizar cualquier movimiento más complejo, como podemos ver en la figura \ref{fig:mov-tortle}.

\begin{figure}[!ht]
	\begin{centering}
		\includegraphics[width=0.8\textwidth]{images/mov-tortle.png}
			\caption{Movimiento de Tortle en forma de cuadrado utilizando unicamente sus instrucciones básicas \texttt{forward} y \texttt{right}.}
				\label{fig:mov-tortle}
	\end{centering}
\end{figure}

Para ilustrar brevemente el lenguaje Logo junto con el robot Turtle, vamos a mostrar como se dibujaría una figura que guarda cierto parecido a una flor.
En el código que se muestra a continuación, creamos una función \texttt{square} que dibuja un cuadrado de tamaño 50 pasos, como podemos ver en la figura \ref{fig:square-turtle}

\begin{lstlisting}
to square
	repeat 4 [forward 50 right 90]
end
\end{lstlisting}

Y a continuación definimos una función \texttt{flower} que utiliza la función \texttt{square} antes descrita y que dibujará una flor, como se puede apreciar en la figura \ref{fig:flower-turtle}

\begin{lstlisting}
to flower
	repeat 36 [right 10 square]
end
\end{lstlisting}

\begin{figure}[!ht]
	\begin{adjustwidth}{\oddsidemargin-1in}{\rightmargin}
	%	\centering
			\begin{subfigure}{\paperwidth}
				\centering
				\includegraphics[scale=.45]{images/square-turtle.png}
				\caption{Cuadrado dibujado por el robot Turtle.}
				\label{fig:square-turtle}
			\end{subfigure}
			\begin{subfigure}{\paperwidth}
				\centering
				\includegraphics[scale=.45]{images/flower-turtle.png}
				\caption{Flor dibujada por el robot Turle.}
				\label{fig:flower-turtle}
			\end{subfigure}
			%\caption{Ejemplo de dibujo utilizando funciones por el robot Turtle.}
		\label{fig:square-flower-turtle}
	\end{adjustwidth}
\end{figure}


El proyecto Tortle ha sido reproducido en proyectos más recientes como \Gls{scratch}\cite{scratch}, el cual se verá en más profundidad en la sección \ref{sec:blockly-scratch}.



\section{Khan Academy}
\label{sec:Khan Academy}

Khan Academy\cite{khan-academy}, con casi 37 millones de usuarios, es uno de los mayores proyectos web para aprender casi de cualquier tema: Matemáticas, estadística, economía o humanidades.


\section{Code.org}
\label{sec:Code.org}

Code.org\cite{code-org} es una plataforma on-line que se dedica exclusivamente a enseñar a programar y cuenta ya con más de 8 millones de usuarios. Cuenta con una gran comunidad de educadores y apoyos entre los que se puede contar al Presidente B. H. Obama. Lleva a cabo proyectos como \emph{Hour of Code} en el que promueve que los usuarios inviertan una hora diaria programando en diferentes juegos, muchos de ellos utilizando \Gls{Blockly}\cite{blockly}.

{\color{red} Parece que estoy haciendo publicidad... shit}


\section{Squeak Etoys}
\label{sec:squeak-etoys}


\section{Blockly y Scratch}
\label{sec:blockly-scratch}


%%%%%%%%%%%%%%%%%%%%%%%%%%%%%%%%%%%%%%%%%%%%%%%%
% 3: Análisis de objetivos y metodología
%%%%%%%%%%%%%%%%%%%%%%%%%%%%%%%%%%%%%%%%%%%%%%%%
\chapter{Análisis de objetivos y metodología}\label{objetivos-metodologia}

\section{Objetivos}
\label{sec:Objetivos}

Este Trabajo Fin de Grado consiste en desarrollar un módulo de simulación de un robot para integrarlo en la plataforma \Gls{descubre}. Para ello se tendrán que cubrir una serie de subobjetivos que nombraremos a continuación.
\begin{itemize}
\item Estudiar el uso y aplicación de diferentes librerías de físicas para generar el robot.
\item Comprensión de la plataforma Descubre así como su posterior modificación.
\item Creación de un simulador de un robot de dos ruedas y su integración en la plataforma Descubre.
\item Modificación del motor de \gls{ijava} y creación de la \acrshort{API} para poder controlar el robot.
\end{itemize}

\section{Metodología}
\label{sec:metodologia}

%%%%%%%%%%%%%%%%%%%%%%%%%%%%%%%%%%%%%%%%%%%%%%%%
% 4: Diseño y resolución del trabajo realizado
%%%%%%%%%%%%%%%%%%%%%%%%%%%%%%%%%%%%%%%%%%%%%%%%
\chapter{Diseño y resolución del trabajo realizado}\label{diseno}




%%%%%%%%%%%%%%%%%%%%%%%%%%%%%%%%%%%%%%%%%%%%%%%%
% 5: Conclusiones y vías futuras
%%%%%%%%%%%%%%%%%%%%%%%%%%%%%%%%%%%%%%%%%%%%%%%%
\chapter{Conclusiones y vías futuras}\label{conslusiones}




% explicar en que se diferencia robode con el resto y porque es mejor o que ventajas tiene
% explicar como resuelvo los problemas que se encuentra un estudiante cuando programa (intro-3º parrafo)

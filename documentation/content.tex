%%%%%%%%%%%%%%%%%%%%%%%%%%%%%%%%%%%%%%%%%%%%%%%%
% 1: Introducción
%%%%%%%%%%%%%%%%%%%%%%%%%%%%%%%%%%%%%%%%%%%%%%%%
\pagenumbering{arabic} % para empezar la numeración con números
\chapter{Introducción}\label{introduccion}

Desde los comienzos de la informática, siempre ha habido una fuerte intención de compartir los conocimientos con el resto de profanos de la informática.
{\color{red}esto lo digo porque yo quiero.}



\section{Aprendizaje usando herramientas informáticas}
\label{sec:aprendizaje}

\subsection{Aprendiendo a programar}
\label{sub:aprendiendo-programar}

\section{Descubre}
\label{sec:descubre}


\section{Motivación y enfoque del proyecto}
\label{sec:motivacion}


%%%%%%%%%%%%%%%%%%%%%%%%%%%%%%%%%%%%%%%%%%%%%%%%
% 2: Estado del arte
%%%%%%%%%%%%%%%%%%%%%%%%%%%%%%%%%%%%%%%%%%%%%%%%
\chapter{Estado del arte}\label{estado-arte}



A nivel global, existen una gran cantidad de proyectos con la única intención de enseñar diferentes aspectos de la informática, como lo es la programación \cite{code-school,code-org,code-academy}, robótica \cite{robomind-web,moway} e incluso \Gls{IA}. 


\section{Robots con movilidad...}
\label{sec:Robots con movilidad...}

\section{Code.org y similares}
\label{sec:Code.org y similares}



%%%%%%%%%%%%%%%%%%%%%%%%%%%%%%%%%%%%%%%%%%%%%%%%
% 3: Análisis de objetivos y metodología
%%%%%%%%%%%%%%%%%%%%%%%%%%%%%%%%%%%%%%%%%%%%%%%%
\chapter{Análisis de objetivos y metodología}\label{objetivos-metodologia}

\section{Objetivos}
\label{sec:Objetivos}

Este Trabajo Fin de Grado consiste en desarrollar un módulo de simulación de un robot para integrarlo en la platforma \Gls{descubre}. Para ello se tendrán que cubrir una serie de subobjetivos que nombraremos a continuación.
\begin{itemize}
\item Estudiar el uso y aplicación de diferentes librerías de físicas para generar el robot.
\item Comprensión de la plataforma Descubre así como su posterior modificación.
\item Creación de un simulador de un robot de dos ruedas y su integración en la plataforma Descubre.
\item Modificación del motor de \gls{ijava} y creación de la \acrshort{API} para poder controlar el robot.
\end{itemize}

\section{Metodología}
\label{sec:metodologia}

%%%%%%%%%%%%%%%%%%%%%%%%%%%%%%%%%%%%%%%%%%%%%%%%
% 4: Diseño y resolución del trabajo realizado
%%%%%%%%%%%%%%%%%%%%%%%%%%%%%%%%%%%%%%%%%%%%%%%%
\chapter{Diseño y resolución del trabajo realizado}\label{diseno}




%%%%%%%%%%%%%%%%%%%%%%%%%%%%%%%%%%%%%%%%%%%%%%%%
% 5: Conclusiones y vías futuras
%%%%%%%%%%%%%%%%%%%%%%%%%%%%%%%%%%%%%%%%%%%%%%%%
\chapter{Conclusiones y vías futuras}\label{conslusiones}

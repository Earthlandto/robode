%   https://es.sharelatex.com/learn/Glossaries

% \gls{ }
% To print the term, lowercase. For example, \gls{maths} prints mathematics when used.
% \Gls{ }
% The same as \gls but the first letter will be printed in uppercase. Example: \Gls{maths} prints Mathematics
% \glspl{ }
% The same as \gls but the term is put in its plural form. For instance, \glspl{formula} will write formulas in your final document.
% \Glspl{ }
% The same as \Gls but the term is put in its plural form. For example, \Glspl{formula} renders as Formulas.
%
% \acrlong{ }
% Displays the phrase which the acronyms stands for. Put the label of the acronym inside the braces. In the example, \acrlong{gcd} prints Greatest Common Divisor.
% \acrshort{ }
% Prints the acronym whose label is passed as parameter. For instance, \acrshort{gcd} renders as GCD.
% \acrfull{ }
% Prints both, the acronym and its definition. In the example the output of \acrfull{lcm} is Least Common Multiple (LCM).

\newacronym{API}{API}{Application Programming Interface}
%\newacronym{js}{JS}{Javascript}


\newglossaryentry{ijava}
{
        name=iJava,
        description={}
}
\newglossaryentry{js}
{
        name=Javascript,
        description={}
}
\newglossaryentry{descubre}
{
        name=Descubre,
        description={}
}
\newglossaryentry{api}
{
        name=API,
        description={}
}
\newglossaryentry{logo}
{
        name=Logo programming,
        description={The Logo Programming Language, a dialect of Lisp, was designed as a tool for learning. Its features - interactivity, modularity, extensibility, flexibility of data types - follow from this goal.}
}
\newglossaryentry{IA}
{
        name=Inteligenaica Artificial,
        description={}
}

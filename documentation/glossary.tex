%   https://es.sharelatex.com/learn/Glossaries

% \gls{ }
% To print the term, lowercase. For example, \gls{maths} prints mathematics when used.
% \Gls{ }
% The same as \gls but the first letter will be printed in uppercase. Example: \Gls{maths} prints Mathematics
% \glspl{ }
% The same as \gls but the term is put in its plural form. For instance, \glspl{formula} will write formulas in your final document.
% \Glspl{ }
% The same as \Gls but the term is put in its plural form. For example, \Glspl{formula} renders as Formulas.
%
% \acrlong{ }
% Displays the phrase which the acronyms stands for. Put the label of the acronym inside the braces. In the example, \acrlong{gcd} prints Greatest Common Divisor.
% \acrshort{ }
% Prints the acronym whose label is passed as parameter. For instance, \acrshort{gcd} renders as GCD.
% \acrfull{ }
% Prints both, the acronym and its definition. In the example the output of \acrfull{lcm} is Least Common Multiple (LCM).



%%%%%%%%%%%%%%%%%%%%%%%%%% ACRONYMS
\newacronym{API}{API}{Application Programming Interface}
\newacronym{MIT}{MIT}{Massachusetts Institute of Technology}
\newacronym{TIC}{TIC}{Tecnologías de la Información y la Comunicaciones}

%%%%%%%%%%%%%%%%%%%%%%%%%% GLOSSARY

\newglossaryentry{ijava}
{
        name=iJava,
        description={}
}
\newglossaryentry{js}
{
        name=Javascript,
        description={}
}
\newglossaryentry{descubre}
{
        name=Descubre,
        description={}
}
\newglossaryentry{api}
{
        name=API,
        description={}
}
\newglossaryentry{logo}
{
        name=Lenguaje Logo,
        description={El lenguaje Logo, basado en el lenguaje Lisp, fue diseñado como una herramienta para aprendizaje. Todas sus características -interactividad, modularidad, extensibilidad, flexibilidad en los tipos de datos- persiguen esta meta}
}
\newglossaryentry{arduino}
{
        name=Arduino,
        description={}
}
\newglossaryentry{turtle}
{
        name=Turtle,
        description={El proyecto más popular del Lenguaje Logo ha evolucionado en la Tortuga, originalmente una criatura robótica que se movia por el suelo siguiendo una serie de instrucciones programadas previamente}
}
\newglossaryentry{lisp}
{
        name=Lisp,
        description={}
}
\newglossaryentry{scratch}
{
        name=Scratch,
        description={}
}
\newglossaryentry{hardware}
{
        name=Hardware,
        description={}
}
\newglossaryentry{software}
{
        name=Software,
        description={}
}
\newglossaryentry{com-euro}
{
        name=Comisión Europea,
        description={Órgano ejecutivo y legislativo de la Unión Europea. Se encarga de proponer la legislación, la aplicación de las decisiones, la defensa de los tratados de la Unión y del día a día de la Union Europe}
}

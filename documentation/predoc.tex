
%%%%%%%%%%%%%%%%%%%%%%%%%%%%%%%%%%%%%%%%%%%%%%%%
% DEDICATORIA
%%%%%%%%%%%%%%%%%%%%%%%%%%%%%%%%%%%%%%%%%%%%%%%%

%\chapter*{}
\pagenumbering{Roman} % para comenzar la numeracion de paginas en numeros romanos
\setcounter{page}{3}

%\begin{flushright}
%\textit{Dedicado a \\
%alguien :-)}
%\end{flushright}


%%%%%%%%%%%%%%%%%%%%%%%%%%%%%%%%%%%%%%%%%%%%%%%%
% AGRADECIMIENTOS
%%%%%%%%%%%%%%%%%%%%%%%%%%%%%%%%%%%%%%%%%%%%%%%%
%\chapter*{Agradecimientos} % si no queremos que añada la palabra "Capitulo"
%\addcontentsline{toc}{chapter}{Agradecimientos} % si queremos que aparezca en el índice
%\markboth{AGRADECIMIENTOS}{AGRADECIMIENTOS} % encabezado





%%%%%%%%%%%%%%%%%%%%%%%%%%%%%%%%%%%%%%%%%%%%%%%%
% RESUMEN
%%%%%%%%%%%%%%%%%%%%%%%%%%%%%%%%%%%%%%%%%%%%%%%%
\chapter*{Resumen} % si no queremos que añada la palabra "Capitulo"
\addcontentsline{toc}{chapter}{Resumen} % si queremos que aparezca en el índice
\markboth{Resumen}{Resumen} % encabezado

En la última década se lleva experimentando un aumento en la aceptación y uso de la informática y tecnología en la sociedad. Servicios como Google, Twitter o Facebook son usados en masa por un gran número de personas. Compañías como Apple o Microsoft son conocidas por la mayoría de los integrantes de la sociedad actual. La informática está presente en cada aspecto de la vida de las personas. Desde investigaciones en el cuidado de la salud, pasando por las relaciones sociales que se establecen y hasta la simple acción de realizar la compra están apoyadas por la informática. La informática tiene un gran papel en nuestra sociedad que cada vez cobra más protagonismo. 

Pero la informática y sus bases siempre han sido algo desconocido para la gran mayoría de personas. Es común encontrar gente que, a pesar de usar diariamente en su trabajo un ordenador, desconoce su funcionamiento. La informática se creó como una herramienta para resolver problemas. No obstante, también es importante que se entienda su funcionamiento, para poder resolver cada vez problemas más complejos.

Desde los primeros años de la informática se ha buscado extender este conocimiento entre los miembros de la sociedad. Especialmente entre los más pequeños, que formarán las nuevas generaciones de informáticos, pero también de médicos, ingenieros y profesores. 

Es imprescindible que se extienda el \emph{pensamiento computacional} entre los miembros más jóvenes de la sociedad. Se ha demostrado en varios estudios que aprender programación en edades tempranas mejora la capacidad de concentración y abstracción, se consigue mayor autonomía y un notable aumento en la creatividad.

Proyectos como Robomind o Scratch buscan justo esto, que alumnos de Educación Primaria y Secundaria desarrollen el \emph{pensamiento computacional}. También existen proyectos como Moway o Lego Mindstorm, que utilizan la robótica para atraer a los más pequeños a aprender jugando. 

Descubre la programación es una plataforma online cuyo objetivo es enseñar a programar a los alumnos de Educación primaria y Secundaria de la Región de Murcia. Descubre proporcionando un editor online, retos y una red social para compartir los programas que se crean y aprender de los compañeros. Otra de sus metas es extender el uso de la plataforma entre los colegios de la Región, permitiendo que los profesores puedan seguir el progreso del alumno mediante diferentes herramientas.

En este proyecto se aglutinan todas estas ideas en una herramienta para aprender a programar. Se ha creado el simulador de un robot que busca hacer la programación una actividad atractiva y entretenida. Este simulador se integrará en la plataforma Descubre la programación, permitiendo así que esté disponible entre los estudiantes de Educación Primaria y Secundaria de la Región de Murcia.

Así, se une la idea de enseñar a programar en un entorno online, a la de usar la robótica como punto de captación y entretenimiento. Aprender a programar jugando. También, el hecho de que sea en un entorno online y gratuito, se pretende conseguir que cualquiera pueda aprender a programar, a cualquier hora y en cualquier lugar.


El robot que se simula tendrá unas características físicas similares a las del robot Moway: dos ruedas que se mueven de manera independiente, un conjunto de sensores externos que detectan colisiones y otro conjunto de sensores inferiores (optorreflectivos) que se usarán para detectar líneas pintadas en el suelo.

De esta manera, se busca que el alumno invente programas que muevan al robot de manera aleatoria, evitando colisiones o siguiendo líneas. Para ello, se proporciona una librería de funciones que permiten controlar el comportamiento del robot. La funcionalidad aportada es básica, buscando de esta manera que los alumnos desarrollen por si mismos comportamientos más complejos.

La finalidad de este proyecto es conseguir que el alumno se interese por la programación desde un punto de vista diferente al que ya existe en Descubre. Que desarrolle su \emph{pensamiento computacional} y que aprenda a programar jugando.


